\documentclass[journal]{IEEEtran}

  \usepackage{graphicx}
  \usepackage{fixltx2e}
  \usepackage{stfloats}
  \usepackage{listings}
  \usepackage{caption}
  
  \usepackage{pgfplots}


\pgfplotsset{
    box plot/.style={
        /pgfplots/.cd,
        black,
        only marks,
        mark=-,
        mark size=0.2em,
        /pgfplots/error bars/.cd,
        y dir=plus,
        y explicit,
    },
    box plot box/.style={
        /pgfplots/error bars/draw error bar/.code 2 args={%
            \draw  ##1 -- ++(0.2em,0pt) |- ##2 -- ++(-0.2em,0pt) |- ##1 -- cycle;
        },
        /pgfplots/table/.cd,
        y index=2,
        y error expr={\thisrowno{3}-\thisrowno{2}},
        /pgfplots/box plot
    },
    box plot top whisker/.style={
        /pgfplots/error bars/draw error bar/.code 2 args={%
            \pgfkeysgetvalue{/pgfplots/error bars/error mark}%
            {\pgfplotserrorbarsmark}%
            \pgfkeysgetvalue{/pgfplots/error bars/error mark options}%
            {\pgfplotserrorbarsmarkopts}%
            \path ##1 -- ##2;
        },
        /pgfplots/table/.cd,
        y index=4,
        y error expr={\thisrowno{2}-\thisrowno{4}},
        /pgfplots/box plot,
    },
    box plot bottom whisker/.style={
        /pgfplots/error bars/draw error bar/.code 2 args={%
            \pgfkeysgetvalue{/pgfplots/error bars/error mark}%
            {\pgfplotserrorbarsmark}%
            \pgfkeysgetvalue{/pgfplots/error bars/error mark options}%
            {\pgfplotserrorbarsmarkopts}%
            \path ##1 -- ##2;
        },
        /pgfplots/table/.cd,
        y index=5,
        y error expr={\thisrowno{3}-\thisrowno{5}},
        /pgfplots/box plot
    },
    box plot median/.style={
        /pgfplots/box plot
    }
}

  

\begin{document}

\title{Project report}

\author{Vincent B\textsc{rillault} \& Pierre P\textsc{fister}}% <-this % stops a space


\markboth{Advanced Computer Networks and Distributed Systems 2012}%
{Shell \MakeLowercase{\textit{et al.}}: Bare Demo of IEEEtran.cls for Journals}
\maketitle





\begin{abstract}
\boldmath
%The abstract goes here.
\end{abstract}
%% IEEEtran.cls defaults to using nonbold math in the Abstract.
%% This preserves the distinction between vectors and scalars. However,
%% if the journal you are submitting to favors bold math in the abstract,
%% then you can use LaTeX's standard command \boldmath at the very start
%% of the abstract to achieve this. Many IEEE journals frown on math
%% in the abstract anyway.
%
%% Note that keywords are not normally used for peerreview papers.
%\begin{IEEEkeywords}
%IEEEtran, journal, \LaTeX, paper, template.
%\end{IEEEkeywords}






\section{Introduction}

\IEEEPARstart{C}{loud} 
% You must have at least 2 lines in the paragraph with the drop letter
% (should never be an issue)
%I wish you the best of success.



%\hfill mds
 
%\hfill January 11, 2007

\subsection{Motivations}





\subsection{Related work}





\subsection{Our work}



\section{Shared memory device}

\subsection{Xen shared memory}



\subsection{Create shared memory}



\subsection{Map memory into user-space}



\subsection{Event channel}



\subsection{Closure}









\section{Shared memory based circular buffer}

\subsection{Principles}



\subsection{Optimizations}



\subsection{Closure and dead-lock avoidance}






\section{Evaluation}


\subsection{Configuration}

\subsection{Throughput}

\begin{figure}[h]
\centering
\begin{tikzpicture}
    \begin{semilogxaxis}[
        xlabel=Buffer size (KiB),
        ylabel=Throughput (Gbps),
        legend style={nodes=right},
        legend pos= north west,
        scaled ticks=false, 
        compat=1.3
        ]
        
    \addplot table [ x index=0,y index=1] {plots/bandwidth.txt};
    \addlegendentry{Xen pipe}
    
    \addplot table[ x index=0,y index=3] {plots/bandwidth.txt};
    \addlegendentry{Stressed Xen pipe}
    
    \addplot table[ x index=0,y index=2] {plots/bandwidth.txt};
    \addlegendentry{TCP}
    
    
    \end{semilogxaxis}
\end{tikzpicture}
\caption{Comparison between Xen pipe and TCP throughputs for different buffer sizes}
\label{shm_size}
\end{figure}


\begin{figure}[h]
\centering
\begin{tikzpicture}
    \begin{semilogxaxis}[
        xlabel=Message size (Bytes),
        ylabel=Throughput (Gbps),
        legend style={nodes=right},
        legend pos= north west,
        scaled ticks=false, 
        compat=1.3
        ]
        
        
        \addplot table[x index=0,y index=4] {plots/messageSize.txt};
    \addlegendentry{XenPipe: 60}
        
      \addplot table[ x index=0,y index=3] {plots/messageSize.txt};
    \addlegendentry{XenPipe: 20}  
        
        
     \addplot table[  x index=0,y index=2] {plots/messageSize.txt};
    \addlegendentry{XenPipe: 5}   
        
    \addplot table[  x index=0,y index=1] {plots/messageSize.txt};
    \addlegendentry{XenPipe: 1}
    
    
    \addplot [smooth,dashed,mark=*,orange] table[  x index=0,y index=5] {plots/messageSize.txt};
    \addlegendentry{XWay}
    
    \addplot [smooth,dashed,mark=square*, violet]  table[  x index=0,y index=7,orange] {plots/messageSize.txt};
    \addlegendentry{XenSocket}

    \addplot [smooth,dashed,mark=x, magenta] table[  x index=0,y index=6] {plots/messageSize.txt};
    \addlegendentry{XenLoop}
   
    \end{semilogxaxis}
\end{tikzpicture}
\caption{Throughput of Xen Pipe with different shared memory sizes (indicated in pages) and other state-of-the-art solutions with respect to the message size}
\label{msg_size}
\end{figure}


\subsection{Delays}

\subsection{Simultaneous transfers}



\begin{figure}[h]
\centering
\begin{tikzpicture}
    \begin{semilogyaxis}[
        xlabel=Number of parallel flows,
        ylabel=Throughput (Mbps),
        legend style={nodes=right},
        legend pos= south west,
        scaled ticks=false, 
        y tick label style={/pgf/number format/.cd,sci,precision=5},
        compat=1.3,
        ]
        
        
        \addplot [blue,mark=x] table[x index=0,y index=1] {plots/multi.txt};
    \addlegendentry{Total throughput}
        
     
    
    \addplot [smooth, red] table[x index=0,y index=8] {plots/multi.txt};
    \addlegendentry{Optimal sharing}
    
     %\addplot [smooth, black] table[x index=0,y index=5] {multi.txt};
    %\addlegendentry{Median value}
    
    %\addplot [smooth, gray] table[x index=0,y index=6] {multi.txt};
    
    %\addplot [smooth, gray] table[x index=0,y index=7] {multi.txt};
       
    %\addplot [smooth, lightgray] table[x index=0,y index=3] {multi.txt};
    
    %\addplot [smooth, lightgray] table[x index=0,y index=4] {multi.txt};
    
    \addplot [box plot median] table {plots/boxes.txt};
    \addplot [box plot box] table {plots/boxes.txt};
    \addplot [box plot top whisker] table {plots/boxes.txt};
    \addplot [box plot bottom whisker] table {plots/boxes.txt};
    \addlegendentry{Values distribution}

    
    \end{semilogyaxis}
\end{tikzpicture}
\caption{Throughputs of multiple parallel flows (60 shared pages per transfert - messages of 32KiB - average value over more than 20 seconds). The dark line corresponds to the median value. 
Gray lines correspond to first and last decile. Light gray lines correspond to the maximum and minimum values}
\label{simult_flows}
\end{figure}





\section{Conclusion}

\subsection{Contributions}


\subsection{Possible improvements}







 
% if have a single appendix:
%\appendix[Proof of the Zonklar Equations]
% or
%\appendix  % for no appendix heading
% do not use \section anymore after \appendix, only \section*
% is possibly needed

% use appendices with more than one appendix
% then use \section to start each appendix
% you must declare a \section before using any
% \subsection or using \label (\appendices by itself
% starts a section numbered zero.)
%


%\appendices
%\section{Figures}
%Appendix one text goes here.




% you can choose not to have a title for an appendix
% if you want by leaving the argument blank

%Appendix two text goes here.


% use section* for acknowledgement
%\section*{Acknowledgment}
%
%
%The authors would like to thank...



% Can use something like this to put references on a page
% by themselves when using endfloat and the captionsoff option.
\ifCLASSOPTIONcaptionsoff
  \newpage
\fi



\begin{thebibliography}{1}

\bibitem{IEEEhowto:kopka}
Xiaolan Zhang et al., \emph{XenSocket: A High-Throughput Interdomain Transport for Virtual Machines}, Middleware '07.

\bibitem{IEEEhowto:kopka}
Kangho Kim et al., \emph{Inter-domain Socket Communications Supporting High Performance and Full Binary Compatibility on Xen}, VEE '08.

\bibitem{IEEEhowto:kopka}
Jian Wang et al., \emph{XenLoop : A Transparent High Performance Inter-VM Network Loopback}, HPDC '08.

\end{thebibliography}





% that's all folks
\end{document}


